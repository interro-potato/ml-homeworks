\documentclass[12pt]{article}
\usepackage[paper=letterpaper,margin=2cm]{geometry}
\usepackage{amsmath}
\usepackage{amssymb}
\usepackage{amsfonts}
\usepackage{newtxtext, newtxmath}
\usepackage{enumitem}
\usepackage{titling}
\usepackage{svg}
\usepackage{xcolor}
\usepackage{listings}
\usepackage{float}
\usepackage{paracol}
\usepackage{nicefrac}
\usepackage{ragged2e}
\usepackage[autostyle]{csquotes}
\usepackage[colorlinks=true]{hyperref}

\MakeOuterQuote{"}
\setlength{\droptitle}{-6em}

\definecolor{codegreen}{rgb}{0,0.6,0}
\definecolor{codegray}{rgb}{0.5,0.5,0.5}
\definecolor{codepurple}{rgb}{0.58,0,0.82}
\definecolor{backcolour}{rgb}{0.95,0.95,0.92}

\lstdefinestyle{mystyle}{
    commentstyle=\color{codegreen},
    keywordstyle=\color{magenta},
    numberstyle=\tiny\color{codegray},
    stringstyle=\color{codepurple},
    basicstyle=\ttfamily\footnotesize,
    breakatwhitespace=false,
    breaklines=true,
    captionpos=b,
    keepspaces=true,
    numbers=left,
    numbersep=5pt,
    showspaces=false,
    showstringspaces=false,
    showtabs=false,
    tabsize=2
}

\lstset{
        style=mystyle,
        inputencoding=utf8,
        extendedchars=true,
}


\title{\large{Aprendizagem 2022}\vskip 0.2cm Homework IV -- Group 019\vskip 0.2cm Diogo Gaspar 99207, Rafael Oliveira 99311}
\date{}
\begin{document}
\maketitle
\center\large{\vskip -2.5cm\textbf{Part I}: Pen and paper}
\begin{enumerate}[leftmargin=\labelsep]

  Given the bivariate observations $\left\{ \begin{bmatrix}
  1.19722 & 1.2214 & 1.24608\\
  1.19722 & 1.2214 & 1.24608\\
\end{bmatrix}, \begin{bmatrix}
  -1\\
  1\\
\end{bmatrix}, \begin{bmatrix}
  1\\
  0\\
\end{bmatrix} \right\}$,
  and the following multivariate Gaussian mixture:

  \begin{equation*}
    \mu_1 = \begin{bmatrix}
  2\\
  2\\
\end{bmatrix}, \quad \mu_2 = \begin{bmatrix}
  0\\
  0\\
\end{bmatrix},
    \quad \Sigma_1 = \begin{bmatrix}
  2 & 1\\
  1 & 2\\
\end{bmatrix}, \quad \Sigma_2 = \begin{bmatrix}
  2 & 0\\
  0 & 2\\
\end{bmatrix},
    \quad \pi_1 = \pi_2 = 0.5
  \end{equation*}

  \item \textbf{Perform one epoch of the EM clustering algorithm and determine the new parameters.}

        As a side note, we'll be using the $k_1$ and $k_2$ notation to represent
        clusters $1$ and $2$ - with that, we'll say that $\pi_1 = P(C = k_1)$,
        with analogous notation for $\pi_2$.

        EM-Clustering, being an unsupervised learning algorithm intending to calculate
        the probability of a sample belonging to a certain cluster, is a method that
        iteratively updates the parameters of the model until convergence is reached
        (for a given definition of convergence). Here, we'll perform exactly one
        epoch of the algorithm, which means we'll be going through two steps:

        \begin{itemize}[leftmargin=\labelsep]
          \item \textbf{E-step:} Here, we're aiming to calculate the \textbf{posterior
                  probability} of each sample belonging to each cluster.
                In order to perform this calculation, we'll be using \textbf{Bayes' rule},
                of course, to decompose the posterior probability into the product of
                the \textbf{likelihood} and the \textbf{prior probability} of the sample
                belonging to the cluster. Let's try, then, to assign each sample to
                the cluster that maximizes the posterior probability.

                For starters, we must first note that the likelihood of a sample
                belonging to a cluster is given by the \textbf{multivariate Gaussian
                  distribution}, which can be written as (considering $d = 2$):

                \begin{equation*}
                  P(x_i \mid C = k_n) \sim \mathcal{N}(x_i; \mu_n, \Sigma_n) = \frac{1}{\sqrt{(2\pi)^d \det \Sigma_n}}
                  \exp \left( -\frac{1}{2} (x - \mu_n)^T \Sigma_n^{-1} (x - \mu_n) \right)
                \end{equation*}

                Moreover, in this step we'll use \textcolor{teal}{teal} to denote the priors and
                \textcolor{purple}{purple} to denote the likelihoods.

                \pagebreak

                As a given, we have that the priors are (for every sample, of course):

                \begin{equation*}
                  \textcolor{teal}{P(C = k_1)} = \textcolor{teal}{P(C = k_2)} = 0.5
                \end{equation*}

                Regarding $x_1$, we have:

                \begin{equation*}
                  \begin{aligned}
                    \textcolor{purple}{P(x_1 \mid C = k_1)}
                     & = \frac{1}{\sqrt{(2\pi)^d \det \Sigma_1}} \exp \left( -\frac{1}{2} (x_1 - \mu_1)^T \Sigma_1^{-1} (x_1 - \mu_1) \right)                                                                                \\
                     & = \frac{1}{\sqrt{(2\pi)^2 \det \begin{bmatrix}
  2 & 1\\
  1 & 2\\
\end{bmatrix}}}
                    \exp \left( -\frac{1}{2} \left(\begin{bmatrix}
  1.19722 & 1.2214 & 1.24608\\
  1.19722 & 1.2214 & 1.24608\\
\end{bmatrix} - \begin{bmatrix}
  2\\
  2\\
\end{bmatrix}\right)^T \begin{bmatrix}
  2 & 1\\
  1 & 2\\
\end{bmatrix}^{-1} \left(\begin{bmatrix}
  1.19722 & 1.2214 & 1.24608\\
  1.19722 & 1.2214 & 1.24608\\
\end{bmatrix} - \begin{bmatrix}
  2\\
  2\\
\end{bmatrix}\right) \right) \\
                     & = 0.0658407
                  \end{aligned}
                \end{equation*}

                \begin{equation*}
                  \begin{aligned}
                    \textcolor{purple}{P(x_1 \mid C = k_2)}
                     & = \frac{1}{\sqrt{(2\pi)^d \det \Sigma_2}} \exp \left( -\frac{1}{2} (x_1 - \mu_2)^T \Sigma_2^{-1} (x_1 - \mu_2) \right)                                                                                \\
                     & = \frac{1}{\sqrt{(2\pi)^2 \det \begin{bmatrix}
  2 & 0\\
  0 & 2\\
\end{bmatrix}}}
                    \exp \left( -\frac{1}{2} \left(\begin{bmatrix}
  1.19722 & 1.2214 & 1.24608\\
  1.19722 & 1.2214 & 1.24608\\
\end{bmatrix} - \begin{bmatrix}
  0\\
  0\\
\end{bmatrix}\right)^T \begin{bmatrix}
  2 & 0\\
  0 & 2\\
\end{bmatrix}^{-1} \left(\begin{bmatrix}
  1.19722 & 1.2214 & 1.24608\\
  1.19722 & 1.2214 & 1.24608\\
\end{bmatrix} - \begin{bmatrix}
  0\\
  0\\
\end{bmatrix}\right) \right) \\
                     & = 0.0227993
                  \end{aligned}
                \end{equation*}


                The (normalized) posteriors can be computed as follows:

                \begin{equation*}
                  \begin{aligned}
                    P(C = k_1 \mid x_1) & = \frac{\textcolor{teal}{P(C = k_1)} \textcolor{purple}{P(x_1 \mid C = k_1)}}{\textcolor{teal}{P(C = k_1)} \textcolor{purple}{P(x_1 \mid C = k_1)} + \textcolor{teal}{P(C = k_2)} \textcolor{purple}{P(x_1 \mid C = k_2)}} \\
                                        & = \frac{0.5 \cdot 0.0658407}{0.5 \cdot 0.0658407 + 0.5 \cdot 0.0227993}                                                                                                                                                    \\
                                        & = 0.742788
                  \end{aligned}
                \end{equation*}

                \begin{equation*}
                  \begin{aligned}
                    P(C = k_2 \mid x_1) & = \frac{\textcolor{teal}{P(C = k_2)} \textcolor{purple}{P(x_1 \mid C = k_2)}}{\textcolor{teal}{P(C = k_1)} \textcolor{purple}{P(x_1 \mid C = k_1)} + \textcolor{teal}{P(C = k_2)} \textcolor{purple}{P(x_1 \mid C = k_2)}} \\
                                        & = \frac{0.5 \cdot 0.0227993}{0.5 \cdot 0.0658407 + 0.5 \cdot 0.0227993}                                                                                                                                                    \\
                                        & = 0.257212
                  \end{aligned}
                \end{equation*}

                Note that, with the aid of the total probability law, we can say that
                $P(C = k_1 \mid x_1) + P(C = k_2 \mid x_1) = 1$; going forward, we'll
                calculate the normalized posterior for $k_2$ utilizing this fact.

                We can now repeat the same process for $x_2$:

                \begin{equation*}
                  \begin{aligned}
                    \textcolor{purple}{P(x_2 \mid C = k_1)}
                     & = \frac{1}{\sqrt{(2\pi)^d \det \Sigma_1}} \exp \left( -\frac{1}{2} (x_2 - \mu_1)^T \Sigma_1^{-1} (x_2 - \mu_1) \right)                                                                                \\
                     & = \frac{1}{\sqrt{(2\pi)^2 \det \begin{bmatrix}
  2 & 1\\
  1 & 2\\
\end{bmatrix}}}
                    \exp \left( -\frac{1}{2} \left(\begin{bmatrix}
  -1\\
  1\\
\end{bmatrix} - \begin{bmatrix}
  2\\
  2\\
\end{bmatrix}\right)^T \begin{bmatrix}
  2 & 1\\
  1 & 2\\
\end{bmatrix}^{-1} \left(\begin{bmatrix}
  -1\\
  1\\
\end{bmatrix} - \begin{bmatrix}
  2\\
  2\\
\end{bmatrix}\right) \right) \\
                     & = 0.00891057
                  \end{aligned}
                \end{equation*}

                \begin{equation*}
                  \begin{aligned}
                    \textcolor{purple}{P(x_2 \mid C = k_2)}
                     & = \frac{1}{\sqrt{(2\pi)^d \det \Sigma_2}} \exp \left( -\frac{1}{2} (x_2 - \mu_2)^T \Sigma_2^{-1} (x_2 - \mu_2) \right)                                                                                \\
                     & = \frac{1}{\sqrt{(2\pi)^2 \det \begin{bmatrix}
  2 & 0\\
  0 & 2\\
\end{bmatrix}}}
                    \exp \left( -\frac{1}{2} \left(\begin{bmatrix}
  -1\\
  1\\
\end{bmatrix} - \begin{bmatrix}
  0\\
  0\\
\end{bmatrix}\right)^T \begin{bmatrix}
  2 & 0\\
  0 & 2\\
\end{bmatrix}^{-1} \left(\begin{bmatrix}
  -1\\
  1\\
\end{bmatrix} - \begin{bmatrix}
  0\\
  0\\
\end{bmatrix}\right) \right) \\
                     & = 0.0482662
                  \end{aligned}
                \end{equation*}

                The (normalized) posteriors can be computed as follows:

                \begin{equation*}
                  \begin{aligned}
                    P(C = k_1 \mid x_2) & = \frac{\textcolor{teal}{P(C = k_1)} \textcolor{purple}{P(x_2 \mid C = k_1)}}{\textcolor{teal}{P(C = k_1)} \textcolor{purple}{P(x_2 \mid C = k_1)} + \textcolor{teal}{P(C = k_2)} \textcolor{purple}{P(x_2 \mid C = k_2)}} \\
                                        & = \frac{0.5 \cdot 0.00891057}{0.5 \cdot 0.00891057 + 0.5 \cdot 0.0482662}                                                                                                                                                  \\
                                        & = 0.155843
                  \end{aligned}
                \end{equation*}

                Like stated above, using the total probability law, we can say that
                $P(C = k_1 \mid x_2) + P(C = k_2 \mid x_2) = 1$; therefore,
                $P(C = k_2 \mid x_2) = 1 - P(C = k_1 \mid x_2) = 0.844157$.

                Finally, repeating the same process for $x_3$:

                \begin{equation*}
                  \begin{aligned}
                    \textcolor{purple}{P(x_2 \mid C = k_1)}
                     & = \frac{1}{\sqrt{(2\pi)^d \det \Sigma_1}} \exp \left( -\frac{1}{2} (x_3 - \mu_1)^T \Sigma_1^{-1} (x_3 - \mu_1) \right)                                                                                \\
                     & = \frac{1}{\sqrt{(2\pi)^2 \det \begin{bmatrix}
  2 & 1\\
  1 & 2\\
\end{bmatrix}}}
                    \exp \left( -\frac{1}{2} \left(\begin{bmatrix}
  1\\
  0\\
\end{bmatrix} - \begin{bmatrix}
  2\\
  2\\
\end{bmatrix}\right)^T \begin{bmatrix}
  2 & 1\\
  1 & 2\\
\end{bmatrix}^{-1} \left(\begin{bmatrix}
  1\\
  0\\
\end{bmatrix} - \begin{bmatrix}
  2\\
  2\\
\end{bmatrix}\right) \right) \\
                     & = 0.0338038
                  \end{aligned}
                \end{equation*}

                \begin{equation*}
                  \begin{aligned}
                    \textcolor{purple}{P(x_3 \mid C = k_2)}
                     & = \frac{1}{\sqrt{(2\pi)^d \det \Sigma_2}} \exp \left( -\frac{1}{2} (x_3 - \mu_2)^T \Sigma_2^{-1} (x_3 - \mu_2) \right)                                                                                \\
                     & = \frac{1}{\sqrt{(2\pi)^2 \det \begin{bmatrix}
  2 & 0\\
  0 & 2\\
\end{bmatrix}}}
                    \exp \left( -\frac{1}{2} \left(\begin{bmatrix}
  1\\
  0\\
\end{bmatrix} - \begin{bmatrix}
  0\\
  0\\
\end{bmatrix}\right)^T \begin{bmatrix}
  2 & 0\\
  0 & 2\\
\end{bmatrix}^{-1} \left(\begin{bmatrix}
  1\\
  0\\
\end{bmatrix} - \begin{bmatrix}
  0\\
  0\\
\end{bmatrix}\right) \right) \\
                     & = 0.061975
                  \end{aligned}
                \end{equation*}

                The (normalized) posteriors can be computed as follows:

                \begin{equation*}
                  \begin{aligned}
                    P(C = k_1 \mid x_3) & = \frac{\textcolor{teal}{P(C = k_1)} \textcolor{purple}{P(x_3 \mid C = k_1)}}{\textcolor{teal}{P(C = k_1)} \textcolor{purple}{P(x_3 \mid C = k_1)} + \textcolor{teal}{P(C = k_2)} \textcolor{purple}{P(x_3 \mid C = k_2)}} \\
                                        & = \frac{0.5 \cdot 0.0338038}{0.5 \cdot 0.0338038 + 0.5 \cdot 0.061975}                                                                                                                                                     \\
                                        & = 0.352936
                  \end{aligned}
                \end{equation*}

                \begin{equation*}
                  P(C = k_2 \mid x_3) = 1 - P(C = k_1 \mid x_3) = 0.647064
                \end{equation*}

          \item \textbf{M-Step: Having calculated the posteriors, we can now
                  update the parameters of the cluster-defining distributions.}

                For each cluster, we'll want to find the new distribution parameters:
                in this case, $\mu_k$ and $\Sigma_k$ (for every cluster $k$).
                For likelihoods, we'll need to update both $\mu_k$ and $\Sigma_k$,
                using all samples weighted by their respective posteriors, as can be seen below;
                for priors, we'll need to perform a weighted mean of the posteriors.

                \begin{equation*}
                  \begin{aligned}
                    \mu_k         & = \frac{\sum_{i=1}^3 P(C = k \mid x_i) x_i}{\sum_{i=1}^3 P(C = k \mid x_i)}                                               \\
                    \Sigma_k^{nm} & = \frac{\sum_{i=1}^3 P(C = k \mid x_i) (x_{i, n} - \mu_{k, n}) (x_{i, m} - \mu_{k, m})^T}{\sum_{i=1}^3 P(C = k \mid x_i)} \\
                    P(C = k)      & = \frac{\sum_{i=1}^3 P(C = k \mid x_i)}{\sum_{c=1}^2\sum_{i=1}^3 P(C = c \mid x_i)}
                  \end{aligned}
                \end{equation*}

                In the equations stated above, we're considering $x_{i, n}$ as the $n$-th
                feature's value of the $i$-th sample, and $\mu_{k, n}$ as the $n$-th
                index of centroid $\mu_k$.

                \pagebreak

                We can now estimate the new parameters of the distributions (and the new
                priors) as can be seen below (note that the new $\mu_k$'s are used
                in the calculation of the new $\Sigma_k$'s):

                % sim rafa, eu sei que não está centrado, por acaso achei que visualmente ficava um efeito fixe

                For $k_1$:

                \begin{equation*}
                  \begin{aligned}
                    \mu_1 & = \frac{\sum_{i=1}^3 P(C = k_1 \mid x_i) x_i}{\sum_{i=1}^3 P(C = k_1 \mid x_i)}                                                                                      \\
                          & = \frac{0.742788 \cdot \begin{bmatrix}
  1.19722 & 1.2214 & 1.24608\\
  1.19722 & 1.2214 & 1.24608\\
\end{bmatrix} + 0.155843 \cdot \begin{bmatrix}
  -1\\
  1\\
\end{bmatrix} + 0.352936 \cdot \begin{bmatrix}
  1\\
  0\\
\end{bmatrix}}{0.742788 + 0.155843 + 0.352936} \\
                          & = \begin{bmatrix}
  0.750964\\
  1.31149\\
\end{bmatrix}
                  \end{aligned}
                \end{equation*}

                \begin{equation*}
                  \begin{aligned}
                    \Sigma_1^{nm} & = \frac{\sum_{i=1}^3 P(C = k_1 \mid x_i) (x_{i, n} - \mu_{k_1, n}) (x_{i, m} - \mu_{k_1, m})^T}{\sum_{i=1}^3 P(C = k_1 \mid x_i)} \\
                    % TODO: check whether it's needed to write intermediate steps here
                                  & = \begin{bmatrix}
  0.436053 & 0.0775726\\
  0.0775726 & 0.778455\\
\end{bmatrix}
                  \end{aligned}
                \end{equation*}

                \begin{equation*}
                  \begin{aligned}
                    \pi_1 = P(C = k_1) = \frac{\sum_{i=1}^3 P(C = k_1 \mid x_i)}{\sum_{c=1}^2\sum_{i=1}^3 P(C = c \mid x_i)} = 0.417189
                    % TODO: once again, check whether it's needed to write intermediate steps here
                  \end{aligned}
                \end{equation*}

                For $k_2$:

                \begin{equation*}
                  \begin{aligned}
                    \mu_2 & = \frac{\sum_{i=1}^3 P(C = k_2 \mid x_i) x_i}{\sum_{i=1}^3 P(C = k_2 \mid x_i)}                                                                                      \\
                          & = \frac{0.257212 \cdot \begin{bmatrix}
  1.19722 & 1.2214 & 1.24608\\
  1.19722 & 1.2214 & 1.24608\\
\end{bmatrix} + 0.844157 \cdot \begin{bmatrix}
  -1\\
  1\\
\end{bmatrix} + 0.647064 \cdot \begin{bmatrix}
  1\\
  0\\
\end{bmatrix}}{0.257212 + 0.844157 + 0.647064} \\
                          & = \begin{bmatrix}
  0.0343846\\
  0.777028\\
\end{bmatrix}
                  \end{aligned}
                \end{equation*}

                \begin{equation*}
                  \begin{aligned}
                    \Sigma_2^{nm} & = \frac{\sum_{i=1}^3 P(C = k_2 \mid x_i) (x_{i, n} - \mu_{k_2, n}) (x_{i, m} - \mu_{k_2, m})^T}{\sum_{i=1}^3 P(C = k_2 \mid x_i)} \\
                    % TODO: "" 
                                  & = \begin{bmatrix}
  0.998818 & -0.215305\\
  -0.215305 & 0.467476\\
\end{bmatrix}
                  \end{aligned}
                \end{equation*}

                \begin{equation*}
                  \begin{aligned}
                    % TODO: ""
                    \pi_2 = P(C = k_2) = \frac{\sum_{i=1}^3 P(C = k_2 \mid x_i)}{\sum_{c=1}^2\sum_{i=1}^3 P(C = c \mid x_i)} = 0.582811
                  \end{aligned}
                \end{equation*}

        \end{itemize}

  \item \textbf{Given the updated parameters computed in previous question:}

        \begin{enumerate}[leftmargin=\labelsep]
          \item \textbf{Perform a hard assignment of observations to clusters under a MAP assumption.}

                Just like in the first question's answer, we'll need to compute the posterior
                probabilities of each sample belonging to each cluster (now utilizing the newly updated
                parameters); however, instead of proceeding to the \textbf{M-Step}, we'll just assign
                each sample to the cluster with the highest posterior probability. Note that,
                since all calculations follow the same formulas utilized in the previous
                question's \textbf{E-Step}, we're not going to repeat them here,
                opting instead to just write the final results.

                The priors have been updated in the previous question's answer to:

                \begin{equation*}
                  \pi_1 = 0.417189, \quad
                  \pi_2 = 0.582811
                \end{equation*}

                Moreover, we've also updated the means and covariances of the distributions
                to:

                \begin{equation*}
                  \begin{aligned}
                    \mu_1 & = \begin{bmatrix}
  0.750964\\
  1.31149\\
\end{bmatrix}       \quad
                          & \Sigma_1                                             & = \begin{bmatrix}
  0.436053 & 0.0775726\\
  0.0775726 & 0.778455\\
\end{bmatrix} \\
                    \mu_2 & = \begin{bmatrix}
  0.0343846\\
  0.777028\\
\end{bmatrix} \quad
                          & \Sigma_2                                             & = \begin{bmatrix}
  0.998818 & -0.215305\\
  -0.215305 & 0.467476\\
\end{bmatrix}
                  \end{aligned}
                \end{equation*}

                % TODO: check whether it's needed to write intermediate steps here

                Therefore, for each sample, we'll have:

                \vspace*{0.5cm}

                \begin{paracol}{3}
                  \setlength{\columnseprule}{1pt}
                  \def\columnseprulecolor{\color{black}}
                  \centering

                  $x_1$:

                  \begin{normalsize}
                    \begin{equation*}
                      \begin{aligned}
                        \textcolor{purple}{P(x_1 | C = k_1)} & = 0.1957  \\
                        \textcolor{purple}{P(x_1 | C = k_2)} & = 0.01352
                      \end{aligned}
                    \end{equation*}

                    \begin{equation*}
                      \begin{aligned}
                        P(C = k_1 | x_1) & = \underline{0.91198} \\
                        P(C = k_2 | x_1) & = 0.088017
                      \end{aligned}
                    \end{equation*}
                  \end{normalsize}

                  \switchcolumn

                  $x_2$:

                  \begin{normalsize}
                    \begin{equation*}
                      \begin{aligned}
                        \textcolor{purple}{P(x_2 | C = k_1)} & = 0.0081953 \\
                        \textcolor{purple}{P(x_2 | C = k_2)} & = 0.14365
                      \end{aligned}
                    \end{equation*}

                    \begin{equation*}
                      \begin{aligned}
                        P(C = k_1 | x_2) & = 0.039237            \\
                        P(C = k_2 | x_2) & = \underline{0.96076}
                      \end{aligned}
                    \end{equation*}
                  \end{normalsize}

                  \switchcolumn

                  $x_3$:

                  \begin{normalsize}
                    \begin{equation*}
                      \begin{aligned}
                        \textcolor{purple}{P(x_3 | C = k_1)} & = 0.077166 \\
                        \textcolor{purple}{P(x_3 | C = k_2)} & = 0.10478
                      \end{aligned}
                    \end{equation*}

                    \begin{equation*}
                      \begin{aligned}
                        P(C = k_1 | x_3) & = 0.34519             \\
                        P(C = k_2 | x_3) & = \underline{0.65481}
                      \end{aligned}
                    \end{equation*}
                  \end{normalsize}

                \end{paracol}

                After performing these calculations, under a MAP (Maximum A Posteriori) assumption,
                we'll assign each sample to the cluster with the highest posterior probability:

                \begin{equation*}
                  \begin{aligned}
                    \text{MAP}(x_1) & \mapsto k_1 \\
                    \text{MAP}(x_2) & \mapsto k_2 \\
                    \text{MAP}(x_3) & \mapsto k_2
                  \end{aligned}
                \end{equation*}

                \pagebreak

          \item \textbf{Compute the silhouette of the larger cluster using the Euclidean distance.}

                As we know, the silhouette of a given sample $x_i$ is defined as:

                \begin{equation*}
                  s_i = \frac{b_i - a_i}{\max(a_i, b_i)}
                \end{equation*}

                where $a_i$ is the average distance between $x_i$ and all other samples
                in the same cluster, and $b_i$ is the average distance between $x_i$ and all
                other samples in its \textbf{neighboring cluster} - the neighboring
                cluster being, therefore, the cluster minimizing such average distance.

                Moreover, the silhouette of a given cluster $k_n$ with $m$ assigned samples is defined as:

                \begin{equation*}
                  s(k_n) = \frac{\sum_{i=1}^m s_i}{m}
                \end{equation*}

                Here, the \textbf{largest cluster} will be the cluster with the
                biggest associated prior value ($\pi_k$). As was computed in 1.,
                $\pi_2 = 0.582811 > 0.417189 = \pi_1$,
                hence the larger cluster will be $k_2$. Its assigned samples,
                considering a MAP assumption, are $x_2$ and $x_3$, so we'll have
                the following:

                \vspace*{0.5cm}

                \begin{paracol}{2}
                  \setlength{\columnseprule}{1pt}
                  \def\columnseprulecolor{\color{black}}
                  \centering

                  $x_2$:

                  \begin{equation*}
                    \begin{aligned}
                      a_2 & = \left\| x_2 - x_3 \right\|_2                                           \\
                          & = \left\| \begin{bmatrix}
  -1\\
  1\\
\end{bmatrix} - \begin{bmatrix}
  1\\
  0\\
\end{bmatrix} \right\|_2
                      = 2.2361                                                                       \\
                      b_2 & = \left\| x_2 - x_1 \right\|_2                                           \\
                          & = \left\| \begin{bmatrix}
  -1\\
  1\\
\end{bmatrix} - \begin{bmatrix}
  1.19722 & 1.2214 & 1.24608\\
  1.19722 & 1.2214 & 1.24608\\
\end{bmatrix} \right\|_2
                      = 2.2361                                                                       \\
                      s_2 & = \frac{b_2 - a_2}{\max(a_2, b_2)}                                       \\
                          & = \frac{2.2361 - 2.2361}{\max(2.2361, 2.2361)} = 0
                    \end{aligned}
                  \end{equation*}

                  \switchcolumn

                  $x_3$:

                  \begin{equation*}
                    \begin{aligned}
                      a_3 & = \left\| x_3 - x_2 \right\|_2                                           \\
                          & = \left\| \begin{bmatrix}
  1\\
  0\\
\end{bmatrix} - \begin{bmatrix}
  -1\\
  1\\
\end{bmatrix} \right\|_2
                      = 2.2361                                                                       \\
                      b_3 & = \left\| x_3 - x_1 \right\|_2                                           \\
                          & = \left\| \begin{bmatrix}
  1\\
  0\\
\end{bmatrix} - \begin{bmatrix}
  1.19722 & 1.2214 & 1.24608\\
  1.19722 & 1.2214 & 1.24608\\
\end{bmatrix} \right\|_2
                      = 2                                                                            \\
                      s_3 & = \frac{b_3 - a_3}{\max(a_3, b_3)}                                       \\
                          & = \frac{2 - 2.2361}{\max(2.2361, 2)} = -0.11803
                    \end{aligned}
                  \end{equation*}

                \end{paracol}

                With this, we can compute the silhouette of the larger cluster:

                \begin{equation*}
                  s(k_2) = \frac{s_2 + s_3}{2} = -0.059015
                \end{equation*}


        \end{enumerate}

\end{enumerate}

\pagebreak

\center\large{\textbf{Part II}: Programming and critical analysis}

\begin{justify}
  The code utilized to answer the following questions is available in this
  report's appendix.
\end{justify}

Recall the \texttt{pd\_speech.arff} dataset from earlier homeworks, centered on
the Parkinson diagnosis from speech features. For the following exercises, normalize
the data using \texttt{sklearn}'s \texttt{MinMaxScaler}.

\begin{enumerate}[leftmargin=\labelsep,resume]

  \item \textbf{Using \texttt{sklearn}, apply $k$-means clustering fully unsupervisedly
          (without targets) on the normalized data with $k = 3$ and three different seeds
          (using $\text{random} \in \{0, 1, 2\}$). Assess the silhouette and purity of the produced solutions.}

  \item \textbf{What is causing the non-determinism?}

  \item \textbf{Using a scatter plot, visualize side-by-side the labeled data using as labels: i) the original
          Parkinson diagnoses, and ii) the previously learned $k = 3$ clusters (\texttt{random = 0}). To this end, select
          the two most informative features as axes and color observations according to their label. For feature
          selection, select the two input variables with highest variance on the MinMax normalized data.}

  \item \textbf{The fraction of variance explained by a principal component is the ratio between the
          variance of that component (i.e., its eigenvalue) and total variance (i.e., sum of all eigenvalues).
          How many principal components are necessary to explain more than 80\% of variability?}

\end{enumerate}

\pagebreak

\large{\textbf{Appendix}\vskip 0.3cm}

\lstinputlisting[language=Python]{code.py}

\end{document}
